\nonstopmode{}
\documentclass[letterpaper]{book}
\usepackage[times,inconsolata,hyper]{Rd}
\usepackage{makeidx}
\makeatletter\@ifl@t@r\fmtversion{2018/04/01}{}{\usepackage[utf8]{inputenc}}\makeatother
% \usepackage{graphicx} % @USE GRAPHICX@
\makeindex{}
\begin{document}
\chapter*{}
\begin{center}
{\textbf{\huge Package `linkSet'}}
\par\bigskip{\large \today}
\end{center}
\ifthenelse{\boolean{Rd@use@hyper}}{\hypersetup{pdftitle = {linkSet: Base Classes for Storing Genomic Link Data}}}{}
\ifthenelse{\boolean{Rd@use@hyper}}{\hypersetup{pdfauthor = {Gilbert Han}}}{}
\begin{description}
\raggedright{}
\item[Title]\AsIs{Base Classes for Storing Genomic Link Data}
\item[Version]\AsIs{0.0.0.1}
\item[Description]\AsIs{This package is created to manage gene-peak links}
\item[License]\AsIs{MIT + file LICENSE}
\item[Encoding]\AsIs{UTF-8}
\item[Roxygen]\AsIs{list(markdown = TRUE)}
\item[RoxygenNote]\AsIs{7.3.2}
\item[Depends]\AsIs{GenomicRanges, SummarizedExperiment}
\item[Imports]\AsIs{methods, Matrix, S4Vectors, IRanges, GenomeInfoDb,
BiocGenerics, Organism.dplyr}
\item[Collate]\AsIs{class.R AllGenerics.R getset.R methods.R annotate.R}
\item[Author]\AsIs{Gilbert Han [aut]}
\item[Maintainer]\AsIs{Gilbert Han }\email{gilberthan1011@gmail.com}\AsIs{}
\end{description}
\Rdcontents{Contents}
\HeaderA{annotatePromoter,linkSet-method}{Annotate the link set with txDb. Give a gene list, and return a}{annotatePromoter,linkSet.Rdash.method}
%
\begin{Description}
Annotate the link set with txDb. Give a gene list, and return a
\end{Description}
%
\begin{Usage}
\begin{verbatim}
## S4 method for signature 'linkSet'
annotatePromoter(x, genome = "hg38", keyType = "symbol", upstream = 500)
\end{verbatim}
\end{Usage}
%
\begin{Arguments}
\begin{ldescription}
\item[\code{x}] linkSet

\item[\code{genome}] the genome you want to annotate

\item[\code{keyType}] the key type. You can check with AnnotationDbi::keytypes

\item[\code{upstream}] The upstream base from the gene
\end{ldescription}
\end{Arguments}
%
\begin{Value}
linkSet object
\end{Value}
\printindex{}
\end{document}
